\documentclass{article}

\usepackage{amsmath}
\usepackage{numprint}

\author{Daniel Fernandes Martins (danielfmt)}
\title{Question \#1 Solution}

\begin{document}

\maketitle

\textbf{Disclaimer.} This is the reasoning I used to solve the problem; it
may be wrong though. This is intended just as food for thought.

\section{Simplifying The Original VC Bound}

The original VC bound looks like this:

\begin{equation*}
\epsilon \leq \sqrt{\frac{8}{N}\ln{\frac{4m_{\mathcal{H}}(2N)}{\delta}}}
\end{equation*}

Putting it in terms of $N$:

\begin{equation*}
N \geq \frac{8}{\epsilon^2}\ln{\frac{4m_{\mathcal{H}}(2N)}{\delta}}
\end{equation*}

Now, defining the growth function $m_{\mathcal{H}}(N)$ in terms of the upper
bound in $d_{vc}$:

\begin{equation*}
N \geq \frac{8}{\epsilon^2}\ln{\frac{4(2N)^{d_{vc}}}{\delta}}
\end{equation*}

\section{Solving Through Successive Approximations}

Starting from $N=10^5$, let's try to find the $N$ that safisfies that
inequality. If we plug in this first $N$ along with $d_{vc}=10$,
$\epsilon=0.05$ and $\delta=0.05$ in the formula, we have:

\begin{equation*}
N \geq \frac{8}{0.05^2}\ln{\frac{4(2\cdot10^5)^{10}}{0.05}} \approx 404,617
\end{equation*}

We've missed by a long shot! If we feed $N=\numprint{404617}$ into the same
equation:

\begin{equation*}
N \geq \frac{8}{0.05^2}\ln{\frac{4(2\cdot404617)^{10}}{0.05}} \approx 449,345
\end{equation*}

Keep doing this long enough so that $N$ converges to approximately
\numprint{452956}.

\end{document}
